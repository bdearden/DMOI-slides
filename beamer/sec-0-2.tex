\documentclass[11pt, compress]{beamer}
\usepackage{amsmath}
\usetheme{Boadilla}
\usepackage[xparse, raster]{tcolorbox}
\tcbset{colback=white, colframe=white}
\NewTColorBox{image}{mmm}{boxrule=0.25pt, colframe=gray, left skip=#1\linewidth,width=#2\linewidth}
\RenewTColorBox{definition}{m}{colback=teal!30!white, colbacktitle=teal!30!white, coltitle=black, colframe=gray, boxrule=0.5pt, sharp corners=downhill, titlerule = 0.25pt, title={#1}}
\RenewTColorBox{theorem}{m}{colback=pink!30!white, colbacktitle=pink!30!white, coltitle=black, colframe=gray, boxrule=0.5pt, sharp corners=downhill, titlerule = 0.25pt, title={#1}}
\RenewTColorBox{proof}{}{boxrule=0.25pt, colframe=gray, colback=white, before upper={Proof:}, after upper={\qed}}
\newcommand{\terminology}[1]{\textbf{#1}}\newcommand{\lt}{<}
\newcommand{\gt}{>}
\newcommand{\amp}{&}

\renewcommand{\d}{\displaystyle}
\newcommand{\N}{\mathbb N}
\newcommand{\B}{\mathbf B}
\newcommand{\Z}{\mathbb Z}
\newcommand{\Q}{\mathbb Q}
\newcommand{\R}{\mathbb R}
\newcommand{\C}{\mathbb C}
\newcommand{\U}{\mathcal U}
\newcommand{\pow}{\mathcal P}
\newcommand{\inv}{^{-1}}
\newcommand{\st}{:}
\renewcommand{\iff}{\leftrightarrow}
\newcommand{\Iff}{\Leftrightarrow}
\newcommand{\imp}{\rightarrow}
\newcommand{\Imp}{\Rightarrow}
\newcommand{\isom}{\cong}

\renewcommand{\bar}{\overline}
\newcommand{\card}[1]{\left| #1 \right|}
\newcommand{\twoline}[2]{\begin{pmatrix}#1 \\ #2 \end{pmatrix}}

\newcommand{\vtx}[2]{node[fill,circle,inner sep=0pt, minimum size=4pt,label=#1:#2]{}}
\newcommand{\va}[1]{\vtx{above}{#1}}
\newcommand{\vb}[1]{\vtx{below}{#1}}
\newcommand{\vr}[1]{\vtx{right}{#1}}
\newcommand{\vl}[1]{\vtx{left}{#1}}
\renewcommand{\v}{\vtx{above}{}}
\title{}
\subtitle{}
\begin{document}
\begin{frame}
\maketitle 
\end{frame}
 
\begin{frame}
\frametitle{Overview}
\tableofcontents 
\end{frame}
 

\section{Warm-up}
\begin{frame}
\frametitle{Warm-up}
 
While walking through a fictional forest, you encounter three trolls guarding a bridge. Each is either a \emph{knight}, who always tells the truth, or a \emph{knave}, who always lies. The trolls will not let you pass until you correctly identify each as either a knight or a knave. Each troll makes a single statement: 
Which troll is which?\end{frame}
 


\section{Atomic and Molecular Statements}
\begin{frame}
\frametitle{Statements: atomic and molecular}
 
\begin{itemize}
\item{}
A \terminology{statement} is any declarative sentence which is either true or false.

\item{}
A statement is \terminology{atomic} if it cannot be divided into smaller statements, otherwise it is called \terminology{molecular}.
\end{itemize}\end{frame}
 
\begin{frame}
\frametitle{}
\begin{example}{}{g:example:idm36}%

These are statements (in fact, \emph{atomic} statements):\begin{itemize}
\item{}
Telephone numbers in the USA have 10 digits.

\item{}
The moon is made of cheese.

\item{}
42 is a perfect square.

\item{}
Every even number greater than 2 can be expressed as the sum of two primes.

\item{}
\(3+7 = 12\)
\end{itemize}\end{example}
\end{frame}
 
\begin{frame}
\frametitle{}
\begin{example}{}{g:example:idm53}%

And these are not statements:\begin{itemize}
\item{}
Would you like some cake?

\item{}
The sum of two squares.

\item{}\(1+3+5+7+\cdots+2n+1\).

\item{}
Go to your room!

\item{}
\(3+x = 12\)
\end{itemize}\end{example}
\end{frame}
 
\begin{frame}
\frametitle{Molecular statements}
 
You can build more complicated (molecular) statements out of simpler (atomic or molecular) ones using \terminology{logical connectives}. For example, this is a molecular statement: \begin{quote}%

Telephone numbers in the USA have 10 digits and 42 is a perfect square.\end{quote}
 
we can break this down into two smaller statements. The two shorter statements are \emph{connected} by an ``and.''\end{frame}
 
\begin{frame}
\frametitle{Logical Connectives}
 
There are five logical connectives we will consider:\pause 
\begin{itemize}[<+->]
\item{}\(P \wedge Q\) is read ``\(P\) and \(Q\),'' and called a \terminology{conjunction}.

\item{}\(P \vee Q\) is read ``\(P\) or \(Q\),'' and called a \terminology{disjunction}.

\item{}\(P \imp Q\) is read ``if \(P\) then \(Q\),'' and called an \terminology{implication} or \terminology{conditional}.

\item{}\(P \iff Q\) is read ``\(P\) if and only if \(Q\),'' and called a \terminology{biconditional}.

\item{}\(\neg P\) is read ``not \(P\),'' and called a \terminology{negation}.
\end{itemize}\end{frame}
 
\begin{frame}
\frametitle{Truth Conditions for Connectives}
 
Key point: the truth-value of a molecular statement is determined by the truth values of its parts and the type of connective.\pause 
\begin{itemize}[<+->]
\item{}\(P \wedge Q\) is true when both \(P\) and \(Q\) are true

\item{}\(P \vee Q\) is true when \(P\) or \(Q\) or both are true.

\item{}\(P \imp Q\) is true when \(P\) is false or \(Q\) is true or both.

\item{}\(P \iff Q\) is true when \(P\) and \(Q\) are both true, or both false.

\item{}\(\neg P\) is true when \(P\) is false.
\end{itemize}\end{frame}
 


\section{Implications}
\begin{frame}
\frametitle{Implications}
 
An \terminology{implication} or \terminology{conditional} is a molecular statement of the form%
\begin{equation*}
P \imp Q
\end{equation*}
where \(P\) and \(Q\) are statements. We say that\begin{itemize}
\item{}\(P\) is the \terminology{hypothesis} (or \terminology{antecedent}).

\item{}\(Q\) is the \terminology{conclusion} (or \terminology{consequent}).
\end{itemize} 
An implication is \emph{true} provided \(P\) is false or \(Q\) is true (or both), and \emph{false} otherwise. In particular, the only way for \(P \imp Q\) to be false is for \(P\) to be true \emph{and} \(Q\) to be false.\end{frame}
 
\begin{frame}
\frametitle{}
\begin{example}{}{g:example:idm163}%

Consider the statement:\begin{quote}%

If Bob gets a 90 on the final, then Bob will pass the class.\end{quote}

\pause 
\begin{itemize}[<+->]
\item{}
This is definitely an implication: \(P\) is the statement ``Bob gets a 90 on the final,'' and \(Q\) is the statement ``Bob will pass the class.''

\item{}
Is it true or false?  What would it take for the statement to be \emph{fasle}?

\item{}
The only way to be false: Bob does get a 90 on the final \emph{AND} Bob still does not pass the class.

\item{}
In particular, if Bob does not get a 90 on the final (\(P\) is false), then whether or not he passes the class, the statement is true.
\end{itemize}\end{example}
\end{frame}
 
\begin{frame}
\frametitle{}
\begin{example}{}{g:example:idm186}%

Decide which of the following statements are true and which are false. Briefly explain.\begin{enumerate}
\item{}
If \(1=1\), then most horses have 4 legs.

\item{}
If \(0=1\), then \(1=1\).

\item{}
If 8 is a prime number, then the 7624th digit of \(\pi\) is an 8.

\item{}
If the 7624th digit of \(\pi\) is an 8, then \(2+2 = 4\).
\end{enumerate}\end{example}
\end{frame}
 
\begin{frame}
\frametitle{Direct Proofs of Implications}
 
The truth conditions for an implication tell us one way we could \emph{prove} that an implication is true. 
\pause 
To prove an implication \(P \imp Q\), it is enough to \emph{assume} \(P\), and from it, \emph{deduce} \(Q\).\end{frame}
 
\begin{frame}
\frametitle{}
\begin{example}{}{g:example:idm215}%

Prove: If two numbers \(a\) and \(b\) are even, then their sum \(a+b\) is even.\end{example}
 \begin{proof}
\pause 
Suppose the numbers \(a\) and \(b\) are even. This means that \(a = 2k\) and \(b=2j\) for some integers \(k\) and \(j\). The sum is then \(a+b = 2k+2j = 2(k+j)\). Since \(k+j\) is an integer, this means that \(a+b\) is even.\end{proof}\end{frame}
 
\begin{frame}
\frametitle{Converse and Contrapositive}
 
How does \(P \imp Q\) relate to \(Q \imp P\)?\pause 
\begin{itemize}[<+->]
\item{}
The \terminology{converse} \index{converse} of an implication \(P \imp Q\) is the implication \(Q \imp P\).

\item{}
The converse is \emph{NOT} logically equivalent to the original implication. That is, whether the converse of an implication is true is independent of the truth of the implication.

\item{}
The \terminology{contrapositive} of an implication \(P \imp Q\) is the statement \(\neg Q \imp \neg P\).

\item{}
An implication and its contrapositive are logically equivalent (they are either both true or both false).
\end{itemize}\end{frame}
 
\begin{frame}
\frametitle{}
\begin{example}{}{g:example:idm256}%

True or false: If you draw any nine playing cards from a regular deck, then you will have at least three cards all of the same suit.
\pause 
Try looking at the contrapositive!
\pause 
Is the converse true?\end{example}
\end{frame}
 
\begin{frame}
\frametitle{}
\begin{example}{}{g:example:idm262}%

Suppose I tell Sue that if she gets a 93\% on her final, then she will get an A in the class. Assuming that what I said is true, what can you conclude in the following cases:
\begin{enumerate}
\item{}
Sue gets a 93\% on her final.

\item{}
Sue gets an A in the class.

\item{}
Sue does not get a 93\% on her final.

\item{}
Sue does not get an A in the class.
\end{enumerate}\end{example}
\end{frame}
 
\begin{frame}
\frametitle{If and only if}
 \begin{quote}%

\(P \iff Q\) is logically equivalent to \((P \imp Q) \wedge (Q \imp P)\).\end{quote}
 
\pause 
Example: Given an integer \(n\), it is true that \(n\) is even if and only if \(n^2\) is even. That is, if \(n\) is even, then \(n^2\) is even, as well as the converse: if \(n^2\) is even, then \(n\) is even.\end{frame}
 
\begin{frame}
\frametitle{}
\begin{example}{}{g:example:idm290}%

Suppose it is true that I sing if and only if I'm in the shower. We know this means both that if I sing, then I'm in the shower, and also the converse, that if I'm in the shower, then I sing. Let \(P\) be the statement, ``I sing,'' and \(Q\) be, ``I'm in the shower.'' So \(P \imp Q\) is the statement ``if I sing, then I'm in the shower.'' Which part of the if and only if statement is this?\end{example}
\end{frame}
 
\begin{frame}
\frametitle{}
\begin{example}{}{g:example:idm300}%

Rephrase the implication, ``if I dream, then I am asleep'' in as many different ways as possible. Then do the same for the converse.\end{example}
\end{frame}
 
\begin{frame}
\frametitle{Necessary and Sufficient}
 
A common way to express the relationship between statements in mathematics is to say that one is necessary or sufficient for the other.\pause 
\begin{itemize}[<+->]
\item{}``\(P\) is necessary for \(Q\)'' means \(Q \imp P\).

\item{}``\(P\) is sufficient for \(Q\)'' means \(P \imp Q\).

\item{}
If \(P\) is necessary and sufficient for \(Q\), then \(P \iff Q\).
\end{itemize}\end{frame}
 
\begin{frame}
\frametitle{}
\begin{example}{}{g:example:idm324}%

Recall from calculus, if a function is differentiable at a point \(c\), then it is continuous at \(c\), but that the converse of this statement is not true (for example, \(f(x) = |x|\) at the point 0). Restate this fact using ``necessary and sufficient'' language.\end{example}
\end{frame}
 


\section{Predicates and Quantifiers}
\begin{frame}
\frametitle{Predicates}
 
\index{free variable}\index{predicate} How could we claim that if \(n\) is prime, then \(n+7\) is not prime? This looks like an implication. I would like to write something like%
\begin{equation*}
P(n) \imp \neg P(n+7)
\end{equation*}
where \(P(n)\) means ``\(n\) is prime.'' But this is not quite right. 
\pause 
For one thing, this sentence has a \terminology{free variable} (that is, a variable that we have not specified anything about), so it is not a statement.  A sentence that contains variables is called a \terminology{predicate}. 
\pause 
If we plug in a specific value for \(n\), we do get a statement. What we really want to say is that \emph{for all} values of \(n\), if \(n\) is prime, then \(n+7\) is not. We need to \emph{quantify} the variable.\end{frame}
 
\begin{frame}
\frametitle{Universal and Existential Quantifiers}
 
The existential quantifier is \(\exists\) and is read ``there exists'' or ``there is.'' For example,%
\begin{equation*}
\exists x (x \lt 0)
\end{equation*}
asserts that there is a number less than 0. 
\pause 
The universal quantifier is \(\forall\) and is read ``for all'' or ``every.'' For example,%
\begin{equation*}
\forall x (x \ge 0)
\end{equation*}
asserts that every number is greater than or equal to 0.\end{frame}
 
\begin{frame}
\frametitle{Quantifiers and Negation}
 
When is a quantified statement false? 
\(\neg \forall x P(x)\) is equivalent to \(\exists x \neg P(x)\). 
\(\neg \exists x P(x)\) is equivalent to \(\forall x \neg P(x)\).\end{frame}
 

\end{document}
