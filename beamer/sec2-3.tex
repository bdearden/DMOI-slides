\documentclass[11pt, compress]{beamer}
\usepackage{amsmath}
\usetheme{Boadilla}
\usefonttheme[onlymath]{serif}
%get rid of navigation:
\setbeamertemplate{navigation symbols}{}


 %%%% Start PreTeXt generated preamble: %%%%% 

%% Some aspects of the preamble are conditional,
%% the LaTeX engine is one such determinant
\usepackage{ifthen}
\newcommand{\tabularfont}{}
\usepackage[xparse, raster]{tcolorbox}
\tcbset{colback=white, colframe=white}
\NewTColorBox{image}{mmm}{boxrule=0.25pt, colframe=gray, left skip=#1\linewidth,width=#2\linewidth}
\RenewTColorBox{definition}{m}{colback=teal!30!white, colbacktitle=teal!30!white, coltitle=black, colframe=gray, boxrule=0.5pt, sharp corners=downhill, titlerule = 0.25pt, title={#1}}
\RenewTColorBox{theorem}{m}{colback=pink!30!white, colbacktitle=pink!30!white, coltitle=black, colframe=gray, boxrule=0.5pt, sharp corners=downhill, titlerule = 0.25pt, title={#1}}
\RenewTColorBox{proof}{}{boxrule=0.25pt, colframe=gray, colback=white, before upper={Proof:}, after upper={\qed}}
\newcommand{\lt}{<}
\newcommand{\gt}{>}
\newcommand{\amp}{&}

%% Begin: Semantic Macros
%% To preserve meaning in a LaTeX file
%%
%% \mono macro for content of "c", "cd", "tag", etc elements
%% Also used automatically in other constructions
%% Simply an alias for \texttt
%% Always defined, even if there is no need, or if a specific tt font is not loaded
\newcommand{\mono}[1]{\texttt{#1}}
%%
%% Following semantic macros are only defined here if their
%% use is required only in this specific document
%%
%% Used for inline definitions of terms
\newcommand{\terminology}[1]{\textbf{#1}}
%% End: Semantic Macros

\renewcommand{\d}{\displaystyle}
\newcommand{\N}{\mathbb N}
\newcommand{\B}{\mathbf B}
\newcommand{\Z}{\mathbb Z}
\newcommand{\Q}{\mathbb Q}
\newcommand{\R}{\mathbb R}
\def\C{\mathbb C}
\def\U{\mathcal U}
\newcommand{\pow}{\mathcal P}
\newcommand{\inv}{^{-1}}
\newcommand{\st}{:}
\renewcommand{\iff}{\leftrightarrow}
\newcommand{\Iff}{\Leftrightarrow}
\newcommand{\imp}{\rightarrow}
\newcommand{\Imp}{\Rightarrow}
\newcommand{\isom}{\cong}

\renewcommand{\bar}{\overline}
\newcommand{\card}[1]{\left| #1 \right|}
\newcommand{\twoline}[2]{\begin{pmatrix}#1 \\ #2 \end{pmatrix}}

\newcommand{\vtx}[2]{node[fill,circle,inner sep=0pt, minimum size=4pt,label=#1:#2]{}}
\newcommand{\va}[1]{\vtx{above}{#1}}
\newcommand{\vb}[1]{\vtx{below}{#1}}
\newcommand{\vr}[1]{\vtx{right}{#1}}
\newcommand{\vl}[1]{\vtx{left}{#1}}
\renewcommand{\v}{\vtx{above}{}}

%% Graphics Preamble Entries
\usepackage{tikz, pgfplots}

\usetikzlibrary{positioning,matrix,arrows}

\usetikzlibrary{shapes,decorations,shadows,fadings,patterns}
\usetikzlibrary{decorations.markings}

\usepackage{skak} %for chessboards etc.

\def\circleA{(-.5,0) circle (1)}
\def\circleAlabel{(-1.5,.6) node[above]{$A$}}
\def\circleB{(.5,0) circle (1)}
\def\circleBlabel{(1.5,.6) node[above]{$B$}}
\def\circleC{(0,-1) circle (1)}
\def\circleClabel{(.5,-2) node[right]{$C$}}
\def\twosetbox{(-2,-1.4) rectangle (2,1.4)}
\def\threesetbox{(-2.5,-2.4) rectangle (2.5,1.4)}
\newcommand{\hexbox}[3]{
  \def\x{-cos{30}*\r*#1+cos{30}*#2*\r*2}
  \def\y{-\r*#1-sin{30}*\r*#1}
  \draw (\x,\y) +(90:\r) -- +(30:\r) -- +(-30:\r) -- +(-90:\r) -- +(-150:\r) -- +(150:\r) -- cycle;
  \draw (\x,\y) node{#3};
}

\tikzset{->-/.style={decoration={
  markings,
  mark=at position .5 with {\arrow{>}}},postaction={decorate}}}

  \newcommand{\onedot}{
    +(.5,.5) \v
  }
  \newcommand{\twodots}{
    +(.25,.25) \v +(.75,.75) \v
  }
  \newcommand{\threedots}{
  +(.25,.25) \v +(.5, .5) \v +(.75,.75) \v
  }
  \newcommand{\fourdots}{
    +(.25,.25) \v +(.25,.75) \v +(.75,.25) \v +(.75,.75) \v
  }
  \newcommand{\fivedots}{
    +(.5,.5) \v +(.25,.25) \v +(.25,.75) \v +(.75,.25) \v +(.75,.75) \v
  }
  \newcommand{\sixdots}{
    +(.25,.5) \v +(.75,.5) \v +(.25,.25) \v +(.25,.75) \v +(.75,.25) \v +(.75,.75) \v
  }
  \newcommand{\dominoborder}{
    \draw[thick, rounded corners] (0,0) rectangle (1,2);
    \draw[thin] (0,1) -- (1,1);
  }


%%%% End of PreTeXt generated preamble %%%%% 

\title{Polynomial Fitting}
\subtitle{(Section 2.3)}
\author{}
\date[]{}

\begin{document}
\begin{frame}
\maketitle 
\end{frame}
 
% \begin{frame}
% \frametitle{Overview}
% \tableofcontents 
% \end{frame}
 
\begin{frame}
\frametitle{Investigate!}
 A standard \(8 \times 8\) chessboard contains 64 squares. Actually, this is just the number of unit squares. How many squares of all sizes are there on a chessboard? Start with smaller boards: \(1\times 1\), \(2 \times 2\), \(3\times 3\), etc. Find a formula for the total number of squares in an \(n\times n\) board.
\end{frame}
 
\begin{frame}
\frametitle{\(\Delta^k\)-constant Sequences}
 Consider the sequence%
\begin{equation*}
1, 5, 14, 30, 55,\ldots
\end{equation*}

 
\pause \vfill 

We can compute the \terminology{sequence of differences}:%
\begin{equation*}
4, 9, 16, 25\ldots
\end{equation*}
and then the \terminology{second differences} (differences of differences):%
\begin{equation*}
5, 7, 9,\ldots
\end{equation*}

 
\pause \vfill 

Eventually some \(k\)-th level of differences \emph{might} be constant (as they are here at the 3rd differences).  Let's call such a sequence a \terminology{\(\Delta^k\)-constant sequence}.
\end{frame}
 
\begin{frame}
\frametitle{}
\begin{example}[2.3.1]Which of the following sequences are \(\Delta^k\)-constant for some value of \(k\)?\begin{enumerate}
\item{} \(2, 3, 7, 14, 24, 37,\ldots\).

\item{} \(1, 8, 27, 64, 125, 216, \ldots\).

\item{} \(1,2,4,8,16,32,64,\ldots\).
\end{enumerate}

\end{example}
\end{frame}
 
\begin{frame}
\frametitle{Finite Differences}
 The closed formula for a sequence will be a degree \(k\) polynomial if and only if the sequence is \(\Delta^k\)-constant (i.e., the \(k\)th sequence of differences is constant).
 
\pause \vfill 

When we know that a sequence can be ``fit'' to a polynomial of a particular degree, we can find the coefficients using the initial terms of the sequence.
\end{frame}
 
\begin{frame}
\frametitle{}
\begin{example}[2.3.2]Find a formula for the sequence \(3, 7, 14, 24,\ldots\). Assume \(a_1 = 3\).
\end{example}
\end{frame}
 
\begin{frame}
\frametitle{}
\begin{example}[2.3.3]Find a closed formula for the number of squares on an \(n \times n\) chessboard.
\end{example}
\end{frame}
 
\begin{frame}
\frametitle{}
\begin{example}[2.3.4]Determine whether the following sequences can be described by a polynomial, and if so, of what degree.\begin{enumerate}
\item{} \(\displaystyle 1, 2, 4, 8, 16, \ldots\)

\item{} \(\displaystyle 0, 7, 50, 183, 484, 1055, \ldots\)

\item{} \(\displaystyle 1,1,2,3,5,8,13,\ldots\)
\end{enumerate}

\end{example}
\end{frame}
 
\end{document}
