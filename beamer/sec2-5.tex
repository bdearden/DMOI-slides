\documentclass[11pt, compress]{beamer}
\usepackage{amsmath}
\usetheme{Boadilla}
\usefonttheme[onlymath]{serif}
%get rid of navigation:
\setbeamertemplate{navigation symbols}{}


 %%%% Start PreTeXt generated preamble: %%%%% 

%% Some aspects of the preamble are conditional,
%% the LaTeX engine is one such determinant
\usepackage{ifthen}
\newcommand{\tabularfont}{}
\usepackage[xparse, raster]{tcolorbox}
\tcbset{colback=white, colframe=white}
\NewTColorBox{image}{mmm}{boxrule=0.25pt, colframe=gray, left skip=#1\linewidth,width=#2\linewidth}
\RenewTColorBox{definition}{m}{colback=teal!30!white, colbacktitle=teal!30!white, coltitle=black, colframe=gray, boxrule=0.5pt, sharp corners=downhill, titlerule = 0.25pt, title={#1}}
\RenewTColorBox{theorem}{m}{colback=pink!30!white, colbacktitle=pink!30!white, coltitle=black, colframe=gray, boxrule=0.5pt, sharp corners=downhill, titlerule = 0.25pt, title={#1}}
\RenewTColorBox{proof}{}{boxrule=0.25pt, colframe=gray, colback=white, before upper={Proof:}, after upper={\qed}}
\newcommand{\lt}{<}
\newcommand{\gt}{>}
\newcommand{\amp}{&}

%% Begin: Semantic Macros
%% To preserve meaning in a LaTeX file
%%
%% \mono macro for content of "c", "cd", "tag", etc elements
%% Also used automatically in other constructions
%% Simply an alias for \texttt
%% Always defined, even if there is no need, or if a specific tt font is not loaded
\newcommand{\mono}[1]{\texttt{#1}}
%%
%% Following semantic macros are only defined here if their
%% use is required only in this specific document
%%
%% End: Semantic Macros

\renewcommand{\d}{\displaystyle}
\newcommand{\N}{\mathbb N}
\newcommand{\B}{\mathbf B}
\newcommand{\Z}{\mathbb Z}
\newcommand{\Q}{\mathbb Q}
\newcommand{\R}{\mathbb R}
\def\C{\mathbb C}
\def\U{\mathcal U}
\newcommand{\pow}{\mathcal P}
\newcommand{\inv}{^{-1}}
\newcommand{\st}{:}
\renewcommand{\iff}{\leftrightarrow}
\newcommand{\Iff}{\Leftrightarrow}
\newcommand{\imp}{\rightarrow}
\newcommand{\Imp}{\Rightarrow}
\newcommand{\isom}{\cong}

\renewcommand{\bar}{\overline}
\newcommand{\card}[1]{\left| #1 \right|}
\newcommand{\twoline}[2]{\begin{pmatrix}#1 \\ #2 \end{pmatrix}}

\newcommand{\vtx}[2]{node[fill,circle,inner sep=0pt, minimum size=4pt,label=#1:#2]{}}
\newcommand{\va}[1]{\vtx{above}{#1}}
\newcommand{\vb}[1]{\vtx{below}{#1}}
\newcommand{\vr}[1]{\vtx{right}{#1}}
\newcommand{\vl}[1]{\vtx{left}{#1}}
\renewcommand{\v}{\vtx{above}{}}

%% Graphics Preamble Entries
\usepackage{tikz, pgfplots}

\usetikzlibrary{positioning,matrix,arrows}

\usetikzlibrary{shapes,decorations,shadows,fadings,patterns}
\usetikzlibrary{decorations.markings}

\usepackage{skak} %for chessboards etc.

\def\circleA{(-.5,0) circle (1)}
\def\circleAlabel{(-1.5,.6) node[above]{$A$}}
\def\circleB{(.5,0) circle (1)}
\def\circleBlabel{(1.5,.6) node[above]{$B$}}
\def\circleC{(0,-1) circle (1)}
\def\circleClabel{(.5,-2) node[right]{$C$}}
\def\twosetbox{(-2,-1.4) rectangle (2,1.4)}
\def\threesetbox{(-2.5,-2.4) rectangle (2.5,1.4)}
\newcommand{\hexbox}[3]{
  \def\x{-cos{30}*\r*#1+cos{30}*#2*\r*2}
  \def\y{-\r*#1-sin{30}*\r*#1}
  \draw (\x,\y) +(90:\r) -- +(30:\r) -- +(-30:\r) -- +(-90:\r) -- +(-150:\r) -- +(150:\r) -- cycle;
  \draw (\x,\y) node{#3};
}

\tikzset{->-/.style={decoration={
  markings,
  mark=at position .5 with {\arrow{>}}},postaction={decorate}}}

  \newcommand{\onedot}{
    +(.5,.5) \v
  }
  \newcommand{\twodots}{
    +(.25,.25) \v +(.75,.75) \v
  }
  \newcommand{\threedots}{
  +(.25,.25) \v +(.5, .5) \v +(.75,.75) \v
  }
  \newcommand{\fourdots}{
    +(.25,.25) \v +(.25,.75) \v +(.75,.25) \v +(.75,.75) \v
  }
  \newcommand{\fivedots}{
    +(.5,.5) \v +(.25,.25) \v +(.25,.75) \v +(.75,.25) \v +(.75,.75) \v
  }
  \newcommand{\sixdots}{
    +(.25,.5) \v +(.75,.5) \v +(.25,.25) \v +(.25,.75) \v +(.75,.25) \v +(.75,.75) \v
  }
  \newcommand{\dominoborder}{
    \draw[thick, rounded corners] (0,0) rectangle (1,2);
    \draw[thin] (0,1) -- (1,1);
  }


%%%% End of PreTeXt generated preamble %%%%% 

\title{Induction}
\subtitle{(Section 2.5)}
\author{}
\date[]{}

\begin{document}
\begin{frame}
\maketitle 
\end{frame}
 
\begin{frame}
\frametitle{Overview}
\tableofcontents 
\end{frame}
 

\section{Recursive Reasoning}
\begin{frame}
\frametitle{Investigate!}
 What is the last digit of \(6^n\) in terms of \(n\)?
 
\pause \vfill 

What if you knew the last digit of \(6^{418}\).  Could you find the last digit of \(6^{419}\)?
 
\pause \vfill 

How can we turn this recursive reasoning into a proof?
\end{frame}
 


\section{Formalizing Proofs}
\begin{frame}
\frametitle{Induction Proof Structure}
 Start by saying what the statement is that you want to prove: ``Let \(P(n)\) be the statement\textellipsis{}'' To prove that \(P(n)\) is true for all \(n \ge 0\), you must prove two facts:\begin{enumerate}
\item{} Base case: Prove that \(P(0)\) is true. You do this directly. This is often easy.


\item{} Inductive case: Prove that \(P(k) \imp P(k+1)\) for all \(k \ge 0\). That is, prove that for any \(k \ge 0\) if \(P(k)\) is true, then \(P(k+1)\) is true as well. This is the proof of an if \textellipsis{} then \textellipsis{} statement, so you can assume \(P(k)\) is true (\(P(k)\) is called the \emph{inductive hypothesis}). \index{inductive hypothesis} You must then explain why \(P(k+1)\) is also true, given that assumption.

\end{enumerate}

 Assuming you are successful on both parts above, you can conclude, ``Therefore by the principle of mathematical induction, the statement \(P(n)\) is true for all \(n \ge 0\).''
\end{frame}
 


\section{Examples}
\begin{frame}
\frametitle{}
\begin{example}[2.5.1]Prove for each natural number \(n \ge 1\) that \(1 + 2 + 3 + \cdots + n = \frac{n(n+1)}{2}\).
\end{example}
\end{frame}
 
\begin{frame}
\frametitle{Proof}
 Let \(P(n)\) be the statement \(1 + 2 + 3 + \cdots + n = \frac{n(n+1)}{2}\). We will show that \(P(n)\) is true for all natural numbers \(n \ge 1\).
 
\pause \vfill 

Base case: \(P(1)\) is the statement \(1 = \frac{1(1+1)}{2}\) which is clearly true.
\end{frame}
 
\begin{frame}
\frametitle{Proof Continued.}
 Inductive case: Let \(k \ge 1\) be a natural number. Assume (for induction) that \(P(k)\) is true. That means \(1 + 2 + 3 + \cdots + k = \frac{k(k+1)}{2}\). We will prove that \(P(k+1)\) is true as well. That is, we must prove that \(1 + 2 + 3 + \cdots + k + (k+1) = \frac{(k+1)(k+2)}{2}\). To prove this equation, start by adding \(k+1\) to both sides of the inductive hypothesis:%
\begin{equation*}
1 + 2 + 3 + \cdots + k + (k+1) = \frac{k(k+1)}{2} + (k+1)\text{.}
\end{equation*}

 
\pause \vfill 

Now, simplifying the right side we get:%
\begin{align*}
\frac{k(k+1)}{2} + k+1 \amp = \frac{k(k+1)}{2} + \frac{2(k+1)}{2}\\
\amp = \frac{k(k+1) + 2(k+1)}{2}\\
\amp = \frac{(k+2)(k+1)}{2}\text{.}
\end{align*}

 
\pause \vfill 

Thus \(P(k+1)\) is true, so by the principle of mathematical induction \(P(n)\) is true for all natural numbers \(n \ge 1\).
\end{frame}
 
\begin{frame}
\frametitle{}
\begin{example}[2.5.2]Prove that for all \(n \in \N\), \(6^n - 1\) is a multiple of 5.
\end{example}
\end{frame}
 
\begin{frame}
\frametitle{}
\begin{example}[2.5.3]Prove that \(n^2 \lt 2^n\) for all integers \(n \ge 5\).
\end{example}
\end{frame}
 


\section{Strong Induction}
\begin{frame}
\frametitle{Chocolate!}
 How many times do you need to break a 4 by 6 chocolate bar to reduce it to single squares?
 
\pause \vfill 

Conjecture: Given a \(n\)-square rectangular chocolate bar, it always takes \(n-1\) breaks to reduce the bar to single squares.
 
\pause \vfill 

Idea: if you break an \(n\)-square bar into two smaller rectangles, of size \(j\) and \(k\), then it takes \(1 + (j-1) + (k-1)\) total breaks.  But this is \(n-1\) breaks!
\end{frame}
 
\begin{frame}
\frametitle{Strong Induction Proof Structure}
 Again, start by saying what you want to prove: ``Let \(P(n)\) be the statement\textellipsis{}'' Then establish two facts:\begin{enumerate}
\item{} Base case: Prove that \(P(0)\) is true.


\item{} Inductive case: Assume \(P(k)\) is true for all \(k \lt n\). Prove that \(P(n)\) is true.

\end{enumerate}
Conclude, ``therefore, by strong induction, \(P(n)\) is true for all \(n \gt 0\).''
\end{frame}
 
\begin{frame}
\frametitle{}
\begin{example}[2.5.5]Prove that any natural number greater than 1 is either prime or can be written as the product of primes.
\end{example}
\end{frame}
 

\end{document}
